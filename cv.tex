\font\titlesize=cmbx10 scaled\magstep2
\font\namesize=cmbx10 scaled\magstep1
\vsize=10in
\hsize=8in
\voffset -0.4in
\hoffset -0.9in
%\tolerance=1000
\vskip 1cm
\centerline{\titlesize CURRICULUM VITAE}
\vskip 0.4in
\centerline{\namesize ALEX McGUIRE PhD}
\vskip 0.2in
\def\jobskip{\noalign{\bigskip}}
\def\posskip{\noalign{\medskip}}
\midinsert
\narrower\narrower\narrower
\noindent I am a senior quantitative analyst and applications designer/developer working for an oil trading company. I have extensive experience in building mathematical models for exotic options in 
different areas of finance, together with the ability to quickly deliver risk applications that meet the needs of front and mid-office. Due to close contact with traders, and my own trading experience, I value
solutions for their practicality and usefulness, rather than any theoretical elegance. I enjoy teaching, and am comfortable communicating my ideas to people with all
levels of experience, including those with very limited knowledge of financial maths and software develpment.
\endinsert


\halign {\indent\bf#\hfil&\quad\vtop{\parindent=0pt\hsize=36em\hangindent.0em\strut#\strut}\cr
\jobskip
{\bf EMPLOYMENT}\cr
\jobskip
Sep 2009 - Date&{\bf Trafigura Ltd}\cr
&2 Portman St\cr
&London W1H 6DU\cr
\posskip
Position:&Head of Risk Systems\cr
\posskip
Duties:&\quad Responsible for rewriting the risk management systems of Trafigura, a commodity trading company. 
The challenge has been to create from scratch a uniform real-time 
reporting application that sits in front of a number of legacy systems, before their eventual replacement. This tool, Starling, has a flexible pivot reporting framework that provides the usual reports, P\&L, P\&L explanation, various greeks etc, as well as permitting intra-day trades and prices/vols to be uploaded seamlessly from traders' Excel blotters.\cr
&\quad Starling was developed rapidly in part due to my team's depth of financial knowledge - none of whom had any previous any financial experience before working with me. The team interacts directly with traders and other end users, rather than relying on business analysts. The adoption of more modern tools is 
another factor, for example the language used for most development was Scala, a hybrid functional/OO language that runs on the JVM and permits more 
rapid development and refactoring.\cr
\jobskip
Oct 2001 -- Jul 2009&{\bf EDF Energy Merchants Ltd}\cr
&Mid City Place\cr
&71 High Holborn\cr
&London WCV 6ED\cr
\posskip
Position:&Senior Quantitative Analyst\cr
\posskip
Duties:&\quad Responsible for modelling the company's portfolio of exotic options. These mostly came about from physical assets, such as
complex swing, storage and hydroelectric deals - including a number of coal power stations that constituted near 10\% of the UK's entire power production. Many of the models required approaches that differed from existing literature, most notably with 
the construction of finely grained price curves from coarse market data, together with new price processes to describe the behaviour of 
these - in particular with respect to price spikes. \cr
&\quad I was greatly involved in the design and development of a risk management system to manage the porfolios that contained these complex deals, 
managing what grew to a team of eight developers. 
Besides being solely responsible for the mathematical side of this system, I designed and implemented large parts of its infrastructure and also developed a profiler, which was critical for the construction of very fast models. \cr
\jobskip
Apr 2000 -- Sep 2001 &{\bf Sempra Energy Trading}\cr
&Tower 42\cr
&25 Old Broad Street\cr
&London EC2N 1HQ\cr
\posskip
Position: &Senior Quantitative Analyst\cr
\posskip
Duties:&\quad This role involved the design of new models for Sempra's growing exotics business, working closely
with the structuring department.\cr
&\quad Along with a small team of developers, I was responsible for the design and development of a new trading/risk management system. 
I left Sempra to follow the head of derivatives to EDF.\cr
\jobskip
Sep 1999 -- March 2000 &{\bf Tokai Bank Europe}\cr
&1 Exchange Square\cr
&London EC2A 2JL\cr
\posskip
Position: &Contractor\cr
\posskip
Duties:&\quad This was a temporary role which involved the analysis of exotic option models being used in a system written in C++. I rewrote most of these models
in order to increase the number of complex deal types that could be handled, while avoiding the problems inherent with the Hull-White approach that had been taken
previously.\cr
\jobskip
Apr 1997 -- Aug 1999&{\bf IBJ International}\cr
&Bracken House\cr
&1 Friday Street\cr
&London EC4M 9JA\cr
\posskip
Position: &Exotic options trader / Quant developer\cr
\posskip
Duties:&\quad At IBJ I made markets in exotic interest rate and FX options, as well as dynamically hedging the portfolio. A large part of my time was spent working on an 
existing exotic options system, written in C++. Here I rewrote most of the models, giving them a more rigorous basis and generalising to a multi-factor setting. These models 
were significantly faster than those offered by commercial systems at the time.\cr
\jobskip
Feb 1997 -- Mar 1997&{\bf Exco Money Broking UK Ltd}\cr
&119 Cannon Street\cr
&London EC4\cr
\posskip
Position: &Contractor\cr
\posskip
Duties:&This was a short temporary position for the development of fixed income tools.  I designed and implemented a real time 'shortest path' algorithm which found arbitrages for the basis swap brokers.\cr
\jobskip
Feb 1995 -- Dec 1996&{\bf Sumitomo Finance International}\cr
&Temple Court\cr
&11 Queen Victoria Street\cr
&London EC4N 4UQ\cr
\posskip
Position: &Options trader / Quant developer\cr
\posskip
Duties:&\quad Originally hired as a junior trader to assist in the market making and hedging of a JGB options book, my increasing interest in finance theory 
led to my running small, yet relatively profitable books which took advantage of various 
inefficiencies in the Japanese and German bond option markets. These inefficiencies included mispriced volatility smiles, incorrect volatility/duration curves and breakdowns 
in the relationships between bond options, repo and futures markets.\cr
\jobskip
April 1993 -- Feb 1995&{\bf Rolfe \& Nolan Plc}\cr
\posskip
Position: &Developer\cr
\posskip
Duties:&\quad I worked on systems for the management of futures and options. \cr
\jobskip
Jan 1992 -- Apr 1993&{\bf Cumulus Systems Ltd}\cr
&1 High Street, Rickmansworth\cr
\posskip
Position: &Developer\cr
\posskip
Duties: &\quad I worked at Cumulus on a back office system for bond trading, writing various reporting tools and client screens.\cr
\noalign{\bigskip}
\noalign{\bigskip}
{\bf FURTHER EDUCATION}\cr
\noalign{\smallskip}
1988--1992&University of Exeter\cr
&PhD Commutative Algebra --- ``On the structure of partially ordered abelian groups with ICC, and integrally closed linearly compact domains''\cr
\noalign{\smallskip}
1984--1988&University of Sheffield\cr
&BSc (Hons) Mathematics I\cr
\noalign{\bigskip\bigskip}
{\bf LANGUAGES}&French (functional)\cr
\noalign{\bigskip\bigskip}
{\bf ADDRESS}&226 Park Rd\cr
&N8 8JX\cr
&London\cr
\noalign{\bigskip}
{\bf PHONE}&020 8348 9464 (Home)\cr
&07885 600778(mobile)\cr
\noalign{\bigskip}
{\bf DATE OF BIRTH}&9 November 1966\cr
\noalign{\bigskip}
}

\bye
