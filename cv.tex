\font\titlesize=cmbx10 scaled\magstep2
\font\namesize=cmbx10 scaled\magstep1
\vsize=10in
\hsize=8in
\voffset -0.4in
\hoffset -0.9in
%\tolerance=1000
\vskip 1cm

\centerline{\titlesize CURRICULUM VITAE}
\vskip 0.4in
\centerline{\namesize ALEX McGUIRE PhD}
\vskip 0.2in
\def\jobskip{\noalign{\bigskip}}
\def\posskip{\noalign{\medskip}}
\midinsert
\narrower\narrower\narrower
I am a senior quantitative analyst and applications designer/developer with extensive experience in building mathematical models for exotic options in 
different areas of finance. Due to close contact with traders, and my own trading experience, I value
solutions for their practicality and usefulness, rather than any theoretical elegance. I enjoy teaching, and am comfortable communicating my ideas to people with all
levels of experience, including those with very limited knowledge of financial maths and software development.

While my recent development work has been in Scala, and my preference is for the functional, 
I am comfortable with different paradigms and have long experience developing in many programming languages
\endinsert


\halign {\indent\bf#\hfil&\quad\vtop{\parindent=0pt\hsize=36em\hangindent.0em\strut#\strut}\cr
\jobskip
{\bf EMPLOYMENT}\cr
\posskip
Jun 2015 - Date&{\bf Topaz Technology Ltd}\cr
\posskip
Position:&{\bf Director/Developer}\cr
\posskip
&\quad Topaz is a startup company developing a risk management and P\&L application, principally in Scala, focussing mostly on the commodities space. The Topaz system is currently in production at Shell and Glencore.\cr
&\quad While developing Topaz I have also provided a wide range of consultancy services for several customers, these include\cr
&$\bullet$ co-ordinating the merger of the IT systems of two downstream oil suppliers\cr
&$\bullet$ implementing a system to optimise constrained LNG charters for long term contracts\cr
&$\bullet$ modelling, valuation and backtesting of gas storage units\cr
&$\bullet$ significantly reducing the time taken by an overnight batch by applying more advanced Monte Carlo techniques\cr
\jobskip

Apr 2021 - Date&{\bf Centrica PLC}\cr
\posskip
Position:&{\bf Quant Analytics}\cr
&\quad Due to a restructuring at Centrica, I switched from being an independent consultant to working on a full-time basis. The nature of the work was unchanged; mostly concerned with building models related to their physical assets, and also architectural redesign of some of their larger risk systems.\cr
\posskip
\jobskip

Sep 2009 - May 2015&{\bf Trafigura Ltd, London UK}\cr
\posskip
Position:&{\bf Head of Risk Systems}\cr
\posskip
Duties:&\quad Responsible for rewriting the risk management systems of Trafigura, a commodity trading company. 
The challenge was to create from scratch a uniform real-time 
reporting application to sit in front of a number of legacy systems, before their eventual replacement. This tool, Starling, had a flexible pivot reporting framework providing the usual reports, P\&L, P\&L explanation, various greeks etc, as well as permitting intra-day trades and prices/vols to be uploaded seamlessly from traders' Excel blotters.\cr
&\quad Starling was written in Scala, and was developed rapidly in part due to my team's depth of financial knowledge - none of whom had any financial experience before working with me. The team interacted directly with traders and other end users, rather than relying on business analysts.\cr
\jobskip
Oct 2001 -- Jul 2009&{\bf EDF Trading Ltd, London}\cr
\posskip
Position:&{\bf Team Lead/Senior Quantitative Analyst}\cr
\posskip
Duties:&\quad Responsible for modelling the company's portfolio of exotic options. These mostly came about from physical assets, such as
complex swing, storage and hydroelectric deals - including a number of coal power stations that constituted near 10\% of the UK's entire power production. Many of the models required approaches that differed from existing literature, most notably with 
the construction of finely grained price curves from coarse market data, together with new price processes to describe the behaviour of 
these - in particular with respect to price spikes. \cr
&\quad My team, which grew to eight developers, developed a risk management system in a mixture of Java, Scala, OCaml and Common Lisp to manage the portfolios that contained these complex deals. \cr
\jobskip
Apr 2000 -- Sep 2001 &{\bf Sempra Energy Trading, London}\cr
\posskip
Position: &{\bf Team Lead/Senior Quantitative Analyst}\cr
\posskip
Duties:&\quad I designed several new models for Sempra's growing exotics business, and advised
the structuring department on technical aspects of proposed deals.\cr
&\quad Leading a small team of developers, I was responsible for the design and development of a new trading/risk management system. 
I left Sempra to follow the head of derivatives to EDF.\cr
\jobskip
\jobskip

Apr 1997 -- Aug 1999&{\bf IBJ International, London}\cr
\posskip
Position: &{\bf Exotic options trader / Quant developer}\cr
\posskip
Duties:&\quad At IBJ I made markets in exotic interest rate and FX options, as well as dynamically hedging the portfolio. I rewrote most of the models in the existing C++ system, 
giving them a more rigorous basis and generalising to a multi-factor setting. These models 
were significantly faster than those offered by commercial systems at the time.\cr
\jobskip
Feb 1995 -- Dec 1996&{\bf Sumitomo Finance International, London}\cr
\posskip
Position: &{\bf Options trader / Quant developer}\cr
\posskip
Duties:&\quad Originally hired as a junior trader to assist in the market making and hedging of a JGB options book, my increasing interest in finance theory 
led to my running small, yet relatively profitable books which took advantage of various 
inefficiencies in the Japanese and German bond option markets. These inefficiencies included mispriced volatility smiles, incorrect volatility/duration curves and breakdowns 
in the relationships between bond options, repo and futures markets.\cr
\noalign{\bigskip}
\noalign{\bigskip}
{\bf FURTHER EDUCATION}\cr
\noalign{\smallskip}
1988--1992&University of Exeter\cr
&PhD Commutative Algebra --- ``On the structure of partially ordered abelian groups with ICC, and integrally closed linearly compact domains''\cr
\noalign{\smallskip}
1984--1988&University of Sheffield\cr
&BSc (Hons) Mathematics I\cr
\noalign{\bigskip\bigskip}
{\bf LANGUAGES}&French (functional)\cr
\noalign{\bigskip}
{\bf ADDRESS}&226 Park Rd, London, N8 8JX\cr
\noalign{\bigskip}
{\bf PHONE} &07885 600778\cr
\noalign{\bigskip}
{\bf DATE OF BIRTH}&9 November 1966\cr
\noalign{\bigskip}
}

\bye
